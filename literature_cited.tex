%!TEX TS-program = lualatex
%!TEX encoding = UTF-8 Unicode

\singlespacing\small

\subsubsection*{Literature Cited}

\begin{hangparas}{1.5em}{1}
	
	1. Goulson, D. 2019. The insect apocalypse, and why it matters. Current Biology 29:R942–R995.
	
	2. Rosenberg, K.V.~et~al. 2019. Decline of the North American avifauna. Science 366:120–124.

	3. Leudtke, J.A.~et~al. 2023. Ongoing declines for the world’s amphibians
in the face of emerging threats. Nature 622:308–314. 
	
	4. Adams, A.M.~et~al. 2024. The state of the bats in North America. Annals of the New York Academy of Sciences 1541:115–128. 
	
	5. Hooper, D.U.~et~al. 2012. A global synthesis reveals biodiversity loss as a major	driver of ecosystem change. Nature 486:105–109.
	
	6. Pimm, S.L. and R.A.\,Askins. 1995. Forest losses predict bird extinctions in eastern North America. Proceedings of the National Academy of Sciences USA 92:9343–9347.
	
%	6. Murphy, J.T. 2021. Globalisation and pollinators: Pollinator declines are an economic threat to global food systems. People and Nature 4:773–785



	Brown, C.\,R., C.\,Baxter, and D.\,N.\,Pashley. 2000. The ecological basis for the conservation of migratory birds in the Mississippi Alluvial Valley. Pages 29–42 \textit{in} R.\,Bonney, D.\,N.\,Pashley, R.\,J.\,Cooper, and L. Niles, editors. Strategies for bird conservation: the Partners in Flight planning process. Proceedings RMRS-P-16. USDA Forest Service, Rocky Mountain Research Station.
	
	
	
	Lapp, S., T.\,Rhinehart, L.\,Freeland-Haynes, J.\,Khilnani, A.\,Syunkova, and J.\,Kitzes. 2023. OpenSoundscape: An open-­source bioacoustics analysis
	package for Python. Methods in Ecolology and Evolution14:2321–2328. 
	
	
	Nelson, P.\,W. 2005. The terrestrial natural communities of Missouri (Revised ed.). Missouri Department of Natural Resources, Jefferson City.
	
	Olson, K.\,R., L.\,W.\,Morton, and D.\,Speidel. 2016\textit{a}. Missouri Ozark Plateau Headwaters Diversion engineering feat. Journal of Soil and Water Conservation 71:13A–19A.
	
	Olson, K.\,R., L.\,W.\,Morton, and D.\,Speidel. 2016\textit{b}. Little River Drainage District conversion of Big Swamp to agricultural land. Journal of Soil and Water Conservation 71:37A–43A.
	
	
	
	Riva, F., and L.\,Fahrig. 2022. The disproportionately high value of small patches for biodiversity conservation. Conservation Letters, 15, e12881. https://doi.org/10.1111/conl.12881
	
	
	Tulloch, A.\,I.\,T., M.\,D.\,Barnes, J.\,Ringma, R.\,A.\,Fuller, and J.\,E.\,M.\,Watson. 2015. Understanding the importance of small patches of habitat for conservation. Journal of Applied Ecology 53:418–429.
	
	Wintle, B.\,A., H.\,Kujala, A.\,Whitehead, A.\,Cameron, S.\,Veloz, A.\,Kukkala, A.\,Moilanen, A.\,Gordon, P.\,E.\,Lentini, N.\,C.\,R.\,Cadenhead, and S.\,A.\,Bekessy. A. 2019. Global synthesis of conservation studies reveals the importance of small habitat patches for biodiversity. Proceedings of the National Academy of Sciences 116:909–914. 
	
	Yan, Y., S.\,Jarvie, Q.\,Zhang, S.\,Zhang, P.\,Han, Q.\,Liu, and P.\,Liu. 2021. Small patches are hotspots for biodiversity conservation in fragmented landscapes. Ecological Indicators 130:108086. 
	
\end{hangparas}


